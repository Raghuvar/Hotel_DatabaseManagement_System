\documentclass[a4,12pt]{report}
\usepackage{graphicx,amsmath,amssymb,amsfonts,listings,color}
\usepackage{color}
\usepackage{hyperref}
\usepackage{graphicx,amsmath,amssymb,amsfonts,listings,color}
\usepackage{geometry}
\usepackage{graphicx,amsmath,amssymb,amsfonts,listings,color}
\usepackage{color}
\usepackage{hyperref}
\usepackage[latin1]{inputenc}
\usepackage{multicol}
\usepackage{calc}
\usepackage{setspace}
\usepackage{fixltx2e}
\usepackage{multicol}
\usepackage[normalem]{ulem}
\geometry{top=10mm, left=30mm, right=30mm, bottom=10mm}
\usepackage{url}

\definecolor{dkgreen}{rgb}{0,0.6,0}
\definecolor{gray}{rgb}{0.5,0.5,0.5}
\definecolor{mauve}{rgb}{0.58,0,0.82}

\lstset{frame=none,
  language=sql,				% whatever language you can choose
  aboveskip=3mm,
  belowskip=3mm,
  showstringspaces=false,
  columns=flexible,
  basicstyle={\small\ttfamily},
  numbers=none,
  numberstyle=\tiny\color{gray},
  keywordstyle=\color{blue},
  commentstyle=\color{dkgreen},
  stringstyle=\color{mauve},
  breaklines=true,
  breakatwhitespace=true,
  tabsize=3
}

\title{ Project Report \\
						\textbf{Databse Management System}\\ Instructor : Prof. PM Jat } 
\author{\textbf{Group$-$7} \\
			Kushal Jangid (201351022)\\
			Raghuvar Prajapati (201351003)\\
			Sameer Bhati (201352013)\\
			Rahul Nalawade (201351017) }
\date{\today}

\begin{document}
\maketitle
\section*{Introduction :} Hotel Management System is a Database Management project for the course CS205 DBMS. Group \#07 has successfully completed the structured design of the HMS considering the efficiency, compactibility, reliability and convenient multi-user accessibility to the system.\\\\
	HMS is not only able to have a massive data storage, but also keeps the data secure and efficiently updates the database. It has been designed according to the requirements of a normal Hotel, providing general guest-services in ambience.\\\\
HMS is developed in such a manner that, it can resolve every data keeping related to a customer (guest). It can store, manipulate, remove and update data in customer's perspective. 


Project Outline:
	Hotel Management System considers a following relations to manage data of customers from the moment he/she enters, till he/she exists.\\

\#Relation ($<$attribute\_list$>$)

\begin{itemize}
\item Guest (guest\_id, entry\_time)\\
\item Family (guest\_id, family\_head\_ssn, address, mobile\_no, no\_of\_adults, no\_of\_children)\\
\item Family\_Member (guest\_id, name, age)\\
\item Company (guest\_id, cname, location)\\
\item Company\_Member (guest\_id, name, age, designation)\\
\item Alloted (guest\_id, room\_no, check\_in\_date, check\_out\_date)\\
\item Room (room\_no, type, rate, status\_occupied)\\
\item Orders (guest\_id, product\_id, date\_of\_orders, time)\\
\item Food (product\_id, type, rate, name)\\
\item Uses (guest\_id, facility\_id, quantity\_used)\\
\item Facility (facility\_id, rate, no\_of\_hours, facility\_type)\\
\item Bill (guest\_id, amount, payment\_date, paying\_method)\\
\end{itemize}
\newpage
\section*{Hotel Management System:} -
This is a real world scenario that we often see in the present world. We have created a database that will take care of the different guest with their information.\\
\subsection*{Relational Schema of the hotel management system:}
Relational Scheme Diagram :\\
\begin{figure}[hbtp]
\centering
\includegraphics[height = 0.6\columnwidth]{Diagram1} \\

\caption{\textbf{{\color{red}Hotel Management System - Relational Schema Diagram}}}
\end{figure}
\newpage
\subsection*{DDL Script :}
This is the all the tables that we have created for the database with all constraint that an attribute is follow to maintain consistency. 
\begin{lstlisting}
DROP SCHEMA hotel CASCADE;
CREATE SCHEMA hotel;

SET SEARCH_PATH TO hotel;

create table Family(
      Guest_ID          	varchar(5) ,
      Family_Head_SSN   	varchar(9) ,
      Address           	varchar(50),
      Mobile_No          	decimal(10,0),
      No_Of_Adults      	smallint,
      No_Of_Children    	smallint,
      primary key(Guest_ID),
      foreign key(Guest_ID) references Guest(Guest_ID)
       ON DELETE CASCADE
);
create table Company(
      Guest_ID              varchar(5) primary key,
      CName                 varchar(40),
      Location              varchar(50),
      foreign key(Guest_ID) references Guest(Guest_ID)
			ON DELETE CASCADE
);

create table Guest(
       Guest_ID       varchar(5) primary key,
       Entry_Time     timestamp
  
);

create table Family_Members(
       Guest_ID            varchar(5),
       Name                varchar(40),
       Age                 smallint,
       primary key(Guest_ID,Name),
       foreign key(Guest_ID) references Family(Guest_ID)
               ON DELETE CASCADE
);

create table Company_Members(
       Guest_ID     varchar(5),
       Name         varchar(40),
       Age          smallint,
       Designation  varchar(40),
       primary key(Name, Guest_ID),
       foreign key(Guest_ID)  references Company(Guest_ID)
                ON DELETE CASCADE
);

create table Room(
       Room_No         varchar(5) primary key,
       Type            varchar(50),
       Rate            int,
       Status_Occupied boolean
);

create table Alloted(
       Guest_ID        varchar(5),
       Room_No         varchar(5),
       Check_in_date   Date,
       Check_out_date  Date,
       primary key(Guest_ID, Room_No),
       foreign key(Guest_ID)  references Guest(Guest_ID)
                ON DELETE CASCADE,     
       foreign key(Room_No) references Room(Room_No)
                ON DELETE CASCADE
);

create table BILL (
       Guest_ID          varchar(5) primary key,
       Amount            int,
       Payment_Date      Date,
       Paying_Method     varchar(20),
       foreign key(Guest_ID)  references Guest(Guest_ID)
                ON DELETE CASCADE
);

CREATE TABLE Facility (
	Facility_id	varchar(5) primary key,
	Rate		int,
	No_of_hours     int,
	Facility_type	varchar(40)
   
);

create table uses (
        Guest_ID       varchar(5),
        Facility_id    varchar(5),
        Quantity_used  int,
        primary key(Guest_ID, Facility_id),
        foreign key(Guest_ID)  references Guest(Guest_ID)
                ON DELETE CASCADE,     
        foreign key(Facility_id) references Facility(Facility_id)
                ON DELETE CASCADE
);

        
create table Food (
        Product_ID   varchar(5) primary key, 
        Rate         int,
        Type         varchar(60),
        Name         varchar(60)
);



create table Orders (
        Guest_ID        varchar(5),
        Product_ID      varchar(5),
        Date_Of_Orders  Date,
        Time            TIME,
        Quantity        smallint,
        primary key(Guest_ID, Product_ID, Date_of_Orders, Time),
        foreign key(Guest_ID)  references Guest(Guest_ID)
                ON DELETE CASCADE,     
        foreign key(Product_ID) references Food(Product_ID)
                ON DELETE CASCADE
);
\end{lstlisting}

\subsection*{ER Diagram :} This is the ER Diagram for database that made relation more understandable.\\
\begin{figure}[hbtp]
\centering
\includegraphics[height = 0.8\columnwidth]{New-ER} \\

\caption{\textbf{{\color{red}Hotel Management System - Relational Schema Diagram}}}
\end{figure}

\newpage
\subsection*{Relations and their Functional Dependencies}


\begin{flushleft}
 \textbf{ Functional Dependencies for all the relations:} \\
 \end{flushleft}


1- \textbf{Guest}(\uline{guest\_id} ,  entity\_time)  \\
		\textbf{FD-} guest\_id $ \rightarrow $ \{ entity\_time  \}\\

2- \textbf{Company}(\uline{guest\_id}, cname, location) \\
	\textbf{FD-} guest\_id $ \rightarrow $ \{ cname \}\\
				\hspace{ 2.00cm }	guest\_id $ \rightarrow $ \{ location \} \\
					
3- \textbf{Company\_Members}(\uline{guest\_id, name}, age, designation )\\
	\textbf{FD-} 		\{ guest\_id , name \}$ \rightarrow $   \{ age \}\\
						\{ guest\_id, name \}$ \rightarrow $ \{ designation \}\\

4- \textbf{Family}(\uline{guest\_id} , head\_ssn , address, mobile\_no. , adults , no\_of\_children ) \\
\textbf{FD-} 	guest\_id $ \rightarrow $ \{ head\_ssn \}\\
				guest\_id $ \rightarrow $ \{ mobile\_no. \}\\
				guest\_id $ \rightarrow $ \{ adults \}\\
				guest\_id $ \rightarrow $ \{ no\_of\_children \}\\


5- \textbf{Family\_Members} (\uline{guest\_id} , \uline{name} , age) \\
\textbf{FD-} \{ guest\_id , name \} $ \rightarrow $ \{ age \}\\

6-	\textbf{Alloted}(\uline{guest\_id} , \uline{room\_id} , check\_in\_date , check\_out\_date ) \\
\textbf{FD-} \{ guest\_id , room\_id \} $ \rightarrow $ \{ check\_in\_date , check\_out\_date \} \\

7-	\textbf{Room} (\uline{room\_no} , type, rate, status\_occupied) \\
\textbf{FD-} room\_no $ \rightarrow $ \{ type, rate, status\_occupied \} \\

8- \textbf{Facility} (facility\_id , no\_of\_hours , rate , type) \\
\textbf{FD-} facility\_id $ \rightarrow $ \{ no\_of\_hours , rate , type \}\\

9- \textbf{Food}(\uline{product\_id} , rate , type , name) \\
\textbf{FD- } product\_id $ \rightarrow $ \{ rate , type , name \} \\

10- \textbf{Uses}(\uline{guest\_id, facility\_id} , quantity\_used) \\
\textbf{FD-} \{ guest\_id , facility\_id \} $ \rightarrow $ \{ quantity\_used \} \\

11- \textbf{Orders}(\uline{guest\_id , product\_id , date\_of\_time , time} , quantity) \\
\textbf{FD-} \{ guest\_id , product\_id , date\_of\_time \} $ \rightarrow $ \{ quantity \}\\

12- \textbf{Bill}(guest\_id , amount, paying\_method, paying\_date ) \\
\textbf{FD-} guest\_id $ \rightarrow $ \{ amount, paying\_method, paying\_date \} \\

\newpage
\subsection*{Relational Algebra AND SQL Queries:}
Here we are giving the queries that our database is supposed to answer:

\textbf{Qu-1} \textbf{  List out the total numbers of vacant rooms ?}\\

\begin{figure}[hbtp]
\centering
\includegraphics[width = 1.1\columnwidth]{ra1}
\caption{\textbf{{\color{red}Relational Algebra Expression}}}
\end{figure}


\textbf{Qu-2} \textbf{ List out the total numbers of filled rooms ? }\\

\begin{figure}[hbtp]
\centering
\includegraphics[width = 1.1\columnwidth]{ra2}
\caption{\textbf{{\color{red}Relational Algebra Expression}}}
\end{figure}

\textbf{Qu-3} \textbf{ List the guest id with their room no, tpye of the room ? }\\

\begin{figure}[hbtp]
\centering
\includegraphics[width = 1.1\columnwidth]{ra3}
\caption{\textbf{{\color{red}Relational Algebra Expression}}}
\end{figure}

\newpage
\textbf{Qu-4} \textbf{ List the guest id who only had food in hotel ? }\\

\begin{figure}[hbtp]
\centering
\includegraphics[width = 1.1\columnwidth]{ra4}
\caption{\textbf{{\color{red}Relational Algebra Expression}}}
\end{figure}

\textbf{Qu-5} \textbf{ Find out the details of the very first customer of the hotel ?}\\

\begin{figure}[hbtp]
\centering
\includegraphics[width = 1.1\columnwidth]{ra5}
\caption{\textbf{{\color{red}Relational Algebra Expression}}}
\end{figure}

\textbf{Qu-6} \textbf{ List out the number of items in each type of food ? }\\

\begin{figure}[hbtp]
\centering
\includegraphics[width = 1.1\columnwidth]{ra6}
\caption{\textbf{{\color{red}Relational Algebra Expression}}}
\end{figure}

\newpage
\textbf{Qu-7} \textbf{ List out the most ordered food items ate by the customers ? }\\

\begin{figure}[hbtp]
\centering
\includegraphics[height = 0.4\columnwidth]{ra7}
\caption{\textbf{{\color{red}Relational Algebra Expression}}}
\end{figure}

\textbf{Qu-8} \textbf{ List out guest id which uses all types of facilities ? }\\

\begin{figure}[hbtp]
\centering
\includegraphics[width = 1.1\columnwidth]{ra8}
\caption{\textbf{{\color{red}Relational Algebra Expression}}}
\end{figure}



\textbf{Qu-10} \textbf{ List out the guest id who were alloted more than or equal to 2 rooms ?}\\

\begin{figure}[hbtp]
\centering
\includegraphics[width = 1.1\columnwidth]{ra10}
\caption{\textbf{{\color{red}Relational Algebra Expression}}}
\end{figure}

\newpage
\textbf{Qu-11} \textbf{ Total Amount paid by the family type customers ? }\\

\begin{figure}[hbtp]
\centering
\includegraphics[width = 1.1\columnwidth]{ra11}
\caption{\textbf{{\color{red}Relational Algebra Expression}}}
\end{figure}

\newpage
\textbf{Qu-12} \textbf{ Total Amount paid by the company type customers ? }\\

\begin{figure}[hbtp]
\centering
\includegraphics[width = 1.1\columnwidth]{ra12}
\caption{\textbf{{\color{red}Relational Algebra Expression}}}
\end{figure}

\textbf{Qu-13} \textbf{ Find out the name of that company that came with maximum number of employee ? }\\

\begin{figure}[hbtp]
\centering
\includegraphics[width = 1.1\columnwidth]{ra13}
\caption{\textbf{{\color{red}Relational Algebra Expression}}}
\end{figure}

\newpage
\textbf{Qu-14} \textbf{ Most Profitable guest id that is of family type customer ? }\\

\begin{figure}[hbtp]
\centering
\includegraphics[width = 1.1\columnwidth]{ra14}
\caption{\textbf{{\color{red}Relational Algebra Expression}}}
\end{figure}

\textbf{Qu-15} \textbf{ Most Profitable guest id that is of company type customer ? }\\

\begin{figure}[hbtp]
\centering
\includegraphics[width = 1.1\columnwidth]{ra15}
\caption{\textbf{{\color{red}Relational Algebra Expression}}}
\end{figure}

\newpage
\textbf{Qu-16} \textbf{ List out the Head SSN with Head Name who came twice to the hotel ? }\\

\begin{figure}[hbtp]
\centering
\includegraphics[width = 1.1\columnwidth]{ra16}
\caption{\textbf{{\color{red}Relational Algebra Expression}}}
\end{figure}

\textbf{Qu-17} \textbf{ List the falility used by more than 10 customers ? }\\

\begin{figure}[hbtp]
\centering
\includegraphics[width = 1.1\columnwidth]{ra17}
\caption{\textbf{{\color{red}Relational Algebra Expression}}}
\end{figure}

\newpage
\textbf{Qu-18} \textbf{ List out the different company names with their number of employees ? }\\

\begin{figure}[hbtp]
\centering
\includegraphics[width = 1.1\columnwidth]{ra18}
\caption{\textbf{{\color{red}Relational Algebra Expression}}}
\end{figure}

\textbf{Qu-19} \textbf{ List out the guest id of the family type with the maximum number of members with them ? }\\

\begin{figure}[hbtp]
\centering
\includegraphics[width = 1.1\columnwidth]{ra19}
\caption{\textbf{{\color{red}Relational Algebra Expression}}}
\end{figure}

\newpage
\textbf{Qu-20} \textbf{ List out the facility used by the most number of customers ? }\\

\begin{figure}[hbtp]
\centering
\includegraphics[width = 1.1\columnwidth]{ra20-1}
\caption{\textbf{{\color{red}}}}
\end{figure}
\begin{figure}[hbtp]
\centering
\includegraphics[width = 1.1\columnwidth]{ra20-2}
\caption{\textbf{{\color{red}Relational Algebra Expression}}}
\end{figure}

\textbf{Qu-21} \textbf{ Find out the total food amount on a particular date given by the customers ? }\\

\begin{figure}[hbtp]
\centering
\includegraphics[width = 0.9\columnwidth]{ra21}
\caption{\textbf{{\color{red}Relational Algebra Expression}}}
\end{figure}

\newpage
\textbf{Qu-22} \textbf{ List out the guest id that orders all types of foods ? }\\

\begin{figure}[hbtp]
\centering
\includegraphics[width = 1.1\columnwidth]{ra22}
\caption{\textbf{{\color{red}Relational Algebra Expression}}}
\end{figure}

\textbf{Qu-23} \textbf{ List out the total number of people who checked out on 11-01-2015 ?}\\

\begin{figure}[hbtp]
\centering
\includegraphics[width = 1.1\columnwidth]{ra23}
\caption{\textbf{{\color{red}Relational Algebra Expression}}}
\end{figure}

\newpage
\textbf{Qu-24} \textbf{ List out the orders for the guest id - C1004 during his period of living ? }\\

\begin{figure}[hbtp]
\centering
\includegraphics[width = 1.0\columnwidth]{ra24}
\caption{\textbf{{\color{red}Relational Algebra Expression}}}
\end{figure}

\textbf{Qu-25} \textbf{ List out the most frequently alloted room ? }\\

\begin{figure}[hbtp]
\centering
\includegraphics[width = 1.1\columnwidth]{ra25}
\caption{\textbf{{\color{red}Relational Algebra Expression}}}
\end{figure}

\newpage
\textbf{Qu-26} \textbf{ Find out the average amount of bill paid by guests visting only for food ? }\\

\begin{figure}[hbtp]
\centering
\includegraphics[width = 1.1\columnwidth]{ra26}
\caption{\textbf{{\color{red}Relational Algebra Expression}}}
\end{figure}

\textbf{Qu-27} \textbf{ Find out the date on which the maximum number customers came to hotel ? }\\

\begin{figure}[hbtp]
\centering
\includegraphics[width = 1.1\columnwidth]{ra27}
\caption{\textbf{{\color{red}Relational Algebra Expression}}}
\end{figure}

\newpage
\textbf{Qu-28} \textbf{ List out the guest id that paid there bills by Cheque?
}\\

\begin{figure}[hbtp]
\centering
\includegraphics[width = 1.0\columnwidth]{ra28}
\caption{\textbf{{\color{red}Relational Algebra Expression}}}
\end{figure}


\textbf{Qu-29} \textbf{ List out the guest id that paid there bills by Cash?
}\\

\begin{figure}[hbtp]
\centering
\includegraphics[width = 1.1\columnwidth]{ra29}
\caption{\textbf{{\color{red}Relational Algebra Expression}}}
\end{figure}

\textbf{Qu-30} \textbf{ List out the guest id that paid there bills by Debit-Card ? }\\

\begin{figure}[hbtp]
\centering
\includegraphics[width = 1.1\columnwidth]{ra30}
\caption{\textbf{{\color{red}Relational Algebra Expression}}}
\end{figure}

\newpage
\textbf{Qu-31} \textbf{ List name, guest id of families and company check in on 12-01-2015 ? }\\

\begin{figure}[hbtp]
\centering
\includegraphics[width = 1.1\columnwidth]{ra31}
\caption{\textbf{{\color{red}Relational Algebra Expression}}}
\end{figure}

\textbf{Qu-32} \textbf{ List out guest id with their room-no. that are of family type ? }\\

\begin{figure}[hbtp]
\centering
\includegraphics[width = 1.1\columnwidth]{ra32}
\caption{\textbf{{\color{red}Relational Algebra Expression}}}
\end{figure}

\newpage
\textbf{Qu-33} \textbf{ List out guest id with their room-no. that are of company type ? }\\

\begin{figure}[hbtp]
\centering
\includegraphics[width = 1.1\columnwidth]{ra33}
\caption{\textbf{{\color{red}Relational Algebra Expression}}}
\end{figure}

\textbf{Qu-34} \textbf{ List out the facility id used by guest that lived in room-no = A101 ? }\\

\begin{figure}[hbtp]
\centering
\includegraphics[width = 1.1\columnwidth]{ra34}
\caption{\textbf{{\color{red}Relational Algebra Expression}}}
\end{figure}

\textbf{Qu-35} \textbf{ List out the date on which maximum number of customer of family tpye came to hotel ? }\\

\begin{figure}[hbtp]
\centering
\includegraphics[width = 1.1\columnwidth]{ra35}
\caption{\textbf{{\color{red}Relational Algebra Expression}}}
\end{figure}

\newpage
\textbf{Qu-36} \textbf{ List out the date on which maximum number of customer of company tpye came to hotel ? }\\

\begin{figure}[hbtp]
\centering
\includegraphics[width = 1.1\columnwidth]{ra36}
\caption{\textbf{{\color{red}Relational Algebra Expression}}}
\end{figure}

\subsection*{SQL Queries:}
\textbf{Qu\_1-} \textbf{List out the total numbers of vacant rooms ?} \\\

\textbf{Ans-}  \textbf{SQl-} \\\

\begin{lstlisting}
			   SELECT		count(room.room_no)
			   FROM			hotel.room 
			   WHERE		room.status_occupied = no;
\end{lstlisting}
			
\textbf{Output-} \\			   
\begin{figure}[hbtp]
\centering
\includegraphics[width = 0.5\columnwidth]{fig1}
\caption{\textbf{{\color{red}Executed output table}}}
\end{figure}


\textbf{Qu\_2-} \textbf{List out the total numbers of filled rooms?}\\\

\textbf{Ans-}		\textbf{SQL-} \\\

\begin{lstlisting}
			   SELECT		count(room.room_no)
			   FROM			hotel.room
			   WHERE		room.status_occupied = yes;
\end{lstlisting}

\textbf{Output-} \\			   
\begin{figure}[hbtp]
\centering
\includegraphics[width = 0.5\columnwidth]{fig2}
\caption{\textbf{{\color{red}Executed output table}}}
\end{figure}

\newpage
\textbf{Qu\_3-}  \textbf{ List the guest\_id with their room\_no, tpye of the room?} \\\

\textbf{Ans-}  \textbf{SQL-} \\\
				
\begin{lstlisting}
			SELECT	alloted.guest_id, 
						alloted.room_no, 
 		 	  			room.type
 		 	  FROM	hotel.room, 
                 	hotel.alloted
           WHERE	alloted.room_no = room.room_no;
\end{lstlisting}
				
\textbf{Output-} \\			   
\begin{figure}[hbtp]
\centering
\includegraphics[width = 0.6\columnwidth]{fig3}
\caption{\textbf{{\color{red}Executed output table}}}
\end{figure}
\\\

\textbf{Qu\_4-}  \textbf{List the guest\_id who only had food in hotel?} \\\

\textbf{Ans-}		\textbf{SQL-} \\\

\begin{lstlisting}
					SELECT	guest.guest_id
					FROM		hotel.guest
				EXCEPT 
					SELECT	alloted.guest_id 
					FROM		hotel.alloted;
\end{lstlisting}

\textbf{Output-} \\			   
\begin{figure}[hbtp]
\centering
\includegraphics[width = 0.3\columnwidth]{fig4}
\caption{\textbf{{\color{red}Executed output table}}}
\end{figure}
\\\

\textbf{Qu\_5-}  \textbf{Find out the details of the very first customer of the hotel?} \\\

\textbf{Ans-}		\textbf{SQL-} \\\
\begin{lstlisting}
					SELECT		.*
					FROM ( SELECT guest.guest_id
							FROM (SELECT min(guest.entry_time) AS m
							FROM hotel.guest ) AS first
							JOIN hotel.guest ON  first.m = guest.entry_time ) AS first1  NATURAL JOIN hotel.family;
\end{lstlisting}

\textbf{Output-} \\			   
\begin{figure}[hbtp]
\centering
\includegraphics[width = 1.2\columnwidth]{fig5}
\caption{\textbf{{\color{red}Executed output table}}}
\end{figure}
\\\

\textbf{Qu\_6-}  \textbf{List out the number of items in each type of food?} \\\

\textbf{Ans-}		\textbf{SQL-} \\\
\begin{lstlisting}
					SELECT	food.type,
								count(food.product_id) AS no_of_items 
					FROM  		hotel.food
					GROUP BY		food.type
\end{lstlisting}
\textbf{Output-} \\			   
\begin{figure}[hbtp]
\centering
\includegraphics[width = 0.4\columnwidth]{fig6}
\caption{\textbf{{\color{red}Executed output table}}}
\end{figure}
\\\
\newpage
\textbf{Qu\_7-}  \textbf{List out the most ordered food items ate by the customers?} \\\

\textbf{Ans-}		\textbf{SQL-} \\\
\begin{lstlisting}
					DROP TABLE item_count;
					CREATE TABLE item_count AS 
					(SELECT	orders.product_id,
							count(orders.guest_id) AS c
					FROM			hotel.orders
					GROUP BY		orders.product_id );
					DROP TABLE  	max_item_count;
					CREATE TABLE  	max_item_count AS 
					(SELECT	max(c) AS  c FROM	item_count );
					DROP TABLE 		max_item;
					CREATE TABLE  	max_item AS
					( SELECT  *
						FROM item_count NATURAL JOIN max_item_count );
					SELECT *
					FROM	hotel.food NATURAL JOIN	max_item
\end{lstlisting}
\textbf{Output-} \\			   
\begin{figure}[hbtp]
\centering
\includegraphics[width = 1.0\columnwidth]{fig7}
\caption{\textbf{{\color{red}Executed output table}}}
\end{figure}
\\\

\textbf{Qu\_8-}  \textbf{List out guest\_id which uses all types of facilities?} \\\

\textbf{Ans-}		\textbf{SQL-} \\\
\begin{lstlisting}
					SELECT  *
					FROM	hotel.guest AS  g
					WHERE NOT EXISTS (( SELECT	f.facility_id 
					FROM	hotel.facility AS  f )
					EXCEPT ( SELECT	u.facility_id
									FROM	hotel.uses AS  u
									WHERE	u.guest_id = g.guest_id
									));
					
\end{lstlisting}
\textbf{Output-} \\			   
\begin{figure}[hbtp]
\centering
\includegraphics[width = 0.7\columnwidth]{fig8}
\caption{\textbf{{\color{red}Executed output table}}}
\end{figure}
\\\

\textbf{Qu\_9-}  \textbf{List guest\_id that lived in room\_no $=$ 'A401' and used facility\_id $=$ 'FC001?} \\\

\textbf{Ans-}		\textbf{SQL-} \\\
\begin{lstlisting}
					SELECT			alloted.room_no, 
										alloted.guest_id,  
										uses.facility_id
					FROM 
										hotel.alloted,
										hotel.uses
					WHERE 			uses.guest_id = alloted.guest_id   AND uses.facility_id = 'FC001' AND alloted.room_no = 'A401';
\end{lstlisting}
\textbf{Output-} \\			   
\begin{figure}[hbtp]
\centering
\includegraphics[width = 0.7\columnwidth]{fig9}
\caption{\textbf{{\color{red}Executed output table}}}
\end{figure}
\\\

\textbf{Qu\_10-}  \textbf{List out the guest\_id who were alloted more than or equal to 2 rooms.?} \\\

\textbf{Ans-}		\textbf{SQL-} \\\
\begin{lstlisting}
					SELECT		alloted.guest_id, 
									count(alloted.room_no) AS no_of_rooms
					FROM				hotel.alloted
            	GROUP BY 		alloted.guest_id
            	HAVING			count( alloted.room_no ) >= 2
            	ORDER BY			alloted.guest_id ;
\end{lstlisting}
\textbf{Output-} \\			   
\begin{figure}[hbtp]
\centering
\includegraphics[width = 0.5\columnwidth]{fig10}
\caption{\textbf{{\color{red}Executed output table}}}
\end{figure}
\\\

\newpage
\textbf{Qu\_11-}  \textbf{Total Amount paid by the family type customers?} \\\

\textbf{Ans-}		\textbf{SQL-} \\\
\begin{lstlisting}
					SELECT 	sum(bill.amount) AS total_family_amount
					FROM		hotel.bill, 
								hotel.family
					WHERE		family.guest_id = bill.guest_id;
\end{lstlisting}
\textbf{Output-} \\			   
\begin{figure}[hbtp]
\centering
\includegraphics[width = 0.3\columnwidth]{fig11}
\caption{\textbf{{\color{red}Executed output table}}}
\end{figure}
\\\

\textbf{Qu\_12-}  \textbf{Total Amount paid by the company type customers?} \\\

\textbf{Ans-}		\textbf{SQL-} \\\
\begin{lstlisting}
					SELECT 	sum(bill.amount) AS total_company_amount
					FROM		hotel.bill, 
								hotel.company
					WHERE		company.guest_id = bill.guest_id;
\end{lstlisting}
\textbf{Output-} \\			   
\begin{figure}[hbtp]
\centering
\includegraphics[width = 0.4\columnwidth]{fig12}
\caption{\textbf{{\color{red}Executed output table}}}
\end{figure}
\\\

\newpage
\textbf{Qu\_13-}  \textbf{Find out the name of that company that came with maximum number of employee?} \\\

\textbf{Ans-}		\textbf{SQL-} \\\
\begin{lstlisting}
					DROP TABLE company_count;
					CREATE TABLE company_count AS ( SELECT	company.name, 
					count(company_members.name) AS no_of_employees
					FROM	hotel.company, 
							hotel.company_members
					WHERE company.guest_id = company_members.guest_id AND company_members.cname = company.name GROUP BY	company.name );
					DROP TABLE max_count;
					CREATE TABLE max_count AS ( SELECT	max(no_of_employees) AS 							no_of_employees FROM company_count );
					SELECT  *
					FROM	max_count NATURAL JOIN	company_count;
\end{lstlisting}
\textbf{Output-} \\			   
\begin{figure}[hbtp]
\centering
\includegraphics[width = 0.6\columnwidth]{fig13}
\caption{\textbf{{\color{red}Executed output table}}}
\end{figure}
\\\

\textbf{Qu\_14-}  \textbf{Most Profitable guest\_id that is of family type customer?} \\\

\textbf{Ans-}		\textbf{SQL-} \\\
\begin{lstlisting}
					DROP TABLE 		family_amount ;
					CREATE TABLE 		family_amount AS (
					SELECT  max(bill.amount) AS amount
					FROM				hotel.bill,
										hotel.family
					WHERE		family.guest_id = bill.guest_id );
					SELECT *
					FROM		hotel.bill NATURAL JOIN	family_amount
\end{lstlisting}	
\textbf{Output-} \\			   
\begin{figure}[hbtp]
\centering
\includegraphics[width = 0.9\columnwidth]{fig14}
\caption{\textbf{{\color{red}Executed output table}}}
\end{figure}
\\\

\textbf{Qu\_15-}  \textbf{Most Profitable guest\_id that is of company type customer?} \\\

\textbf{Ans-}		\textbf{SQL-} \\\
\begin{lstlisting}
					DROP TABLE company_amount ;
					CREATE TABLE comnpany_amount AS ( SELECT  max(bill.amount)
					AS amount
					FROM				hotel.bill, 
										hotel.company
					WHERE		company.guest_id = bill.guest_id );
					SELECT  *
					FROM	hotel.bill NATURAL JOIN	company_amount
\end{lstlisting}			
\textbf{Output-} \\			   
\begin{figure}[hbtp]
\centering
\includegraphics[width = 0.9\columnwidth]{fig15}
\caption{\textbf{{\color{red}Executed output table}}}
\end{figure}
\\\

\textbf{Qu\_16-}  \textbf{List out the Head\_SSN with Head\_Name who came twice to the hotel?} \\\

\textbf{Ans-}		\textbf{SQL-} \\\
\begin{lstlisting}
					SELECT DISTINCT		f1.family_head_ssn, 
												f1.family_head_name,  
												count(f1.guest_id) 
					FROM					hotel.family f1, 
											hotel.family f2  
					WHERE			f1.family_head_ssn = f2.family_head_ssn AND 								f1.guest_id != f2.guest_id 
					GROUP BY			f1.family_head_ssn,
										f1.family_head_name 
					HAVING		count(f1.guest_id) = 2
\end{lstlisting}
\textbf{Output-} \\			   
\begin{figure}[hbtp]
\centering
\includegraphics[width = 0.8\columnwidth]{fig16}
\caption{\textbf{{\color{red}Executed output table}}}
\end{figure}
\\\

\newpage
\textbf{Qu\_17-}  \textbf{List the facility used by more than 10 customers?} \\\

\textbf{Ans-}		\textbf{SQL-} \\\
\begin{lstlisting}
					SELECT				uses.facility_id,
										count(uses.guest_id) AS  c,
										facility.facility_type
					FROM				hotel.facility, 
										hotel.uses
					WHERE			uses.facility_id = facility.facility_id
					GROUP BY		uses.facility_id,
										facility.facility_type 
					HAVING		count(uses.guest_id) >= 10
					
\end{lstlisting}
\textbf{Output-} \\			   
\begin{figure}[hbtp]
\centering
\includegraphics[width = 0.7\columnwidth]{fig17}
\caption{\textbf{{\color{red}Executed output table}}}
\end{figure}
\\\

\textbf{Qu\_18-}  \textbf{List out the different company names with their number of employees?} \\\

\textbf{Ans-}		\textbf{SQL-} \\\
\begin{lstlisting}
					SELECT				company.cname,
										count(company_members.name)
					FROM				hotel.company_members,
										hotel.company
					WHERE		company_members.guest_id = company.guest_id 
					GROUP BY		company.cname
\end{lstlisting}						
\textbf{Output-} \\			   
\begin{figure}[hbtp]
\centering
\includegraphics[width = 0.5\columnwidth]{fig18}
\caption{\textbf{{\color{red}Executed output table}}}
\end{figure}
\\\


\textbf{Qu\_19-}  \textbf{List out the guest\_id of the family type with the maximum number of members with them?} \\\

\textbf{Ans-}		\textbf{SQL-} \\\
\begin{lstlisting}
					DROP TABLE 			f_members; 
					CREATE TABLE 		f_members AS ( SELECT	family.guest_id,
											sum(family.no_of_adults family.no_of_children) AS no_of_members
					FROM		hotel.family  GROUP BY	family.guest_id );
					DROP TABLE 		f1_members;
					CREATE TABLE 	f1_members AS ( SELECT	max(no_of_members)  AS 											no_of_members FROM	f_members );
					SELECT	*
					FROM		f_members NATURAL  JOIN	f1_members
\end{lstlisting}	
\textbf{Output-} \\			   
\begin{figure}[hbtp]
\centering
\includegraphics[width = 0.7\columnwidth]{fig19}
\caption{\textbf{{\color{red}Executed output table}}}
\end{figure}
\\\


\textbf{Qu\_20-}  \textbf{List out the facility used by the most number of customers?} \\\

\textbf{Ans-}		\textbf{SQL-} \\\
\begin{lstlisting}
					DROP TABLE 		facility_used;
					CREATE TABLE 	facility_used AS ( SELECT facility.facility_type, 
																			facility.facility_id, 
																			count(uses.guest_id) AS facility_count
										FROM			hotel.facility, 
														hotel.uses
										WHERE 		uses.facility_id = facility.facility_id
					GROUP BY					facility.facility_id,
												facility.facility_type );
					DROP TABLE 			max_facility_used;
					CREATE TABLE 		max_facility_used AS ( SELECT	max(facility_count) AS facility_count FROM		facility_used );
					SELECT  *
					FROM	max_facility_used  NATURAL JOIN	facility_used
\end{lstlisting}					
\textbf{Output-} \\			   
\begin{figure}[hbtp]
\centering
\includegraphics[width = 1.1\columnwidth]{fig20}
\caption{\textbf{{\color{red}Executed output table}}}
\end{figure}
\\\



\newpage	
\textbf{Qu\_21-}  \textbf{Find out the total food amount on a particular date given by the customers?} \\\

\textbf{Ans-}		\textbf{SQL-} \\\
\begin{lstlisting}
					SELECT				orders.date_of_orders,
											sum(food.rate * orders.quantity) 
					FROM				hotel.orders,
										hotel.food
					WHERE				food.product_id = orders.product_id
					GROUP BY			orders.date_of_orders 
					ORDER BY   		orders.date_of_orders
\end{lstlisting}					
\textbf{Output-} \\			   
\begin{figure}[hbtp]
\centering
\includegraphics[height=0.7\columnwidth]{fig21}
\caption{\textbf{{\color{red}Executed output table}}}
\end{figure}
\\\

\newpage				
\textbf{Qu\_22-}  \textbf{List out the guest\_id that orders all types of foods?} \\\

\textbf{Ans-}		\textbf{SQL-} \\\
\begin{lstlisting}
					SELECT 	* 
					FROM		hotel.guest AS  g
					WHERE NOT EXISTS ( ( SELECT		food.type 
													FROM		hotel.food  ) 
				EXCEPT ( SELECT			food.type 
								FROM			hotel.orders 
					JOIN  hotel.food	ON orders.product_id = food.product_id             
					WHERE	orders.guest_id = g.guest_id  
					 ) ); 
\end{lstlisting}					 
\textbf{Output-} \\			   
\begin{figure}[hbtp]
\centering
\includegraphics[width = 1.0\columnwidth]{fig22}
\caption{\textbf{{\color{red}Executed output table}}}
\end{figure}
\\\

\newpage
\textbf{Qu\_23-}  \textbf{List out the total number of people who checked out on 11-01-2015 ?} \\\

\textbf{Ans-}		\textbf{SQL-} \\\
\begin{lstlisting}
					SELECT DISTINCT		alloted.check_out_date, 
												alloted.guest_id
					FROM						hotel.alloted, 
												hotel.company,
												hotel.family
					WHERE (company.guest_id = allot.guest_id OR family.guest_id = allot.guest_id) AND 	allot.check_out_date = '2015-01-11';
\end{lstlisting}
\textbf{Output-} \\			   
\begin{figure}[hbtp]
\centering
\includegraphics[width = 0.8\columnwidth]{fig23}
\caption{\textbf{{\color{red}Executed output table}}}
\end{figure}
\\\


\textbf{Qu\_24-}  \textbf{List out the orders for the guest\_id - C1004 during his period of living ?} \\\

\textbf{Ans-}		\textbf{SQL-} \\\

\begin{lstlisting}
					SELECT			orders.guest_id, 
										orders.product_id,  
										orders.date_of_orders, 
										orders."time", 
										orders.quantity
					FROM		hotel.orders
					WHERE		orders.guest_id = 'C1004'
\end{lstlisting}						
\textbf{Output-} \\			   
\begin{figure}[hbtp]
\centering
\includegraphics[height=0.4\columnwidth]{fig24}
\caption{\textbf{{\color{red}Executed output table}}}
\end{figure}
\\\



\newpage
\textbf{Qu\_25-}  \textbf{List out the most frequently alloted room ?} \\\

\textbf{Ans-}		\textbf{SQL-}
\begin{lstlisting}
						DROP TABLE frequency1;
						CREATE TABLE frequency1 AS ( SELECT alloted.room_no,
						count(alloted.guest_id) as frequency
						FROM 	hotel.alloted
						GROUP BY	alloted.room_no
						ORDER BY	count(allot.guest_id) );
						DROP TABLE frequency2;
						CREATE TABLE frequency2 AS ( SELECT	max(frequency) AS 				frequency	FROM	frequency1 );
						SELECT	*
						FROM	frequency1	NATURAL JOIN	frequency2
\end{lstlisting}
\textbf{Output-} \\			   
\begin{figure}[hbtp]
\centering
\includegraphics[width=0.4\columnwidth]{fig25}
\caption{\textbf{{\color{red}Executed output table}}}
\end{figure}
\\\

\textbf{Qu\_26-}  \textbf{Find out the average amount of bill paid by guests visting only for food?} \\\

\textbf{Ans-}		\textbf{SQL-}
\begin{lstlisting}
					DROP TABLE 		o_food;
					CREATE TABLE o_food AS ( SELECT guest.guest_id FROM 														hotel.guest	
					EXCEPT	
					SELECT 	allot.guest_id
					FROM 	hotel.allot 	);
					DROP TABLE o1_food;
					CREATE TABLE o1_food AS ( SELECT *	FROM	hotel.bill NATURAL JOIN	o_food );      
					SELECT	sum(o1_food.amount)/count(o1_food.guest_id) AS average
					FROM	o1_food;
\end{lstlisting}
\textbf{Output-} \\			   
\begin{figure}[hbtp]
\centering
\includegraphics[width=0.2\columnwidth]{fig26}
\caption{\textbf{{\color{red}Executed output table}}}
\end{figure}
\\\

\textbf{Qu\_27-}  \textbf{Find out the date on which the maximum number customers came to hotel?} \\\

\textbf{Ans-}		\textbf{SQL-}
\begin{lstlisting}
					DROP TABLE new;
					CREATE TABLE new AS (
					SELECT 	count( guest.guest_id) AS no_of_customers,
										date(guest.entry_time)
					FROM 	hotel.guest
					GROUP BY	date(guest.entry_time) );
					DROP TABLE new1;
					CREATE TABLE new1 AS (
					SELECT	max(no_of_customers) AS no_of_customers
										FROM	new );
					SELECT	*
					FROM	new	NATURAL JOIN	new1
\end{lstlisting}
\textbf{Output-} \\			   
\begin{figure}[hbtp]
\centering
\includegraphics[width=0.7\columnwidth]{fig27}
\caption{\textbf{{\color{red}Executed output table}}}
\end{figure}
\\\

\textbf{Qu\_28-}  \textbf{List out the guest\_id that paid there bills by Cheque?} \\\

\textbf{Ans-}		\textbf{SQL-}
\begin{lstlisting}
					SELECT	bill.guest_id, 
								bill.bill_no, 
  		  						bill.paying_method
					FROM 
		  						hotel.bill
					WHERE		bill.paying_method = 'By-Cheque'
		  			ORDER BY	bill.guest_id
\end{lstlisting}
\textbf{Output-} \\			   
\begin{figure}[hbtp]
\centering
\includegraphics[width=0.4\columnwidth]{fig28}
\caption{\textbf{{\color{red}Executed output table}}}
\end{figure}
\\\

\textbf{Qu\_29-}  \textbf{List out the guest\_id that paid there bills by Cash?} \\\

\textbf{Ans-}		\textbf{SQL-}
\begin{lstlisting}
					SELECT	bill.guest_id, 
								bill.bill_no, 
  		  						bill.paying_method
					FROM 
		  						hotel.bill
					WHERE		bill.paying_method = 'By-Cash'
		  			ORDER BY	bill.guest_id
\end{lstlisting}
\textbf{Output-} \\			   
\begin{figure}[hbtp]
\centering
\includegraphics[width=0.4\columnwidth]{fig29}
\caption{\textbf{{\color{red}Executed output table}}}
\end{figure}
\\\

\textbf{Qu\_30-}  \textbf{List out the guest\_id that paid there bills by Debit-Card?} \\\

\textbf{Ans-}		\textbf{SQL-}
\begin{lstlisting}
					SELECT	bill.guest_id, 
								bill.bill_no, 
  		  						bill.paying_method
					FROM 
		  						hotel.bill
					WHERE		bill.paying_method = 'By-Debit Card'
		  			ORDER BY	bill.guest_id
\end{lstlisting}
\textbf{Output-} \\			   
\begin{figure}[hbtp]
\centering
\includegraphics[width=0.4\columnwidth]{fig30}
\caption{\textbf{{\color{red}Executed output table}}}
\end{figure}
\\\

\textbf{Qu\_31-}  \textbf{List name, guest\_id of families and company check in on 12-01-2015?} \\\

\textbf{Ans-}		\textbf{SQL-}
\begin{lstlisting}
					SELECT DISTINCT		alloted.guest_id, 
  		  								alloted.check_in_date, 
 		  								family.family_head_ssn AS head_ssn_Cname
					FROM 				hotel.alloted, 
  										hotel.family
					WHERE 			(family.guest_id = allot.guest_id ) AND  								allot.check_in_date = '2015-01-12' 	UNION
					SELECT DISTINCT  	allot.guest_id, 
		  									allot.check_in_date, 
 		 									company.name
					FROM 					hotel.allot, 
		 	 								hotel.company
					WHERE 			(company.guest_id = alloted.guest_id ) AND  								alloted.check_in_date = '2015-01-12'
\end{lstlisting}
\textbf{Output-} \\			   
\begin{figure}[hbtp]
\centering
\includegraphics[width=0.6\columnwidth]{fig31}
\caption{\textbf{{\color{red}Executed output table}}}
\end{figure}
\\\

\textbf{Qu\_32-}  \textbf{List out guest id with their room-no. that are of family type?} \\\

\textbf{Ans-}		\textbf{SQL-}
\begin{lstlisting}
					SELECT 		 	family.guest_id, 
 		  								alloted.room_no
					FROM 				hotel.alloted, 
		  								hotel.family
					WHERE 			family.guest_id = alloted.guest_id;
\end{lstlisting}
\textbf{Output-} \\			   
\begin{figure}[hbtp]
\centering
\includegraphics[width=0.5\columnwidth]{fig32}
\caption{\textbf{{\color{red}Executed output table}}}
\end{figure}
\\\

\textbf{Qu\_33-}  \textbf{List out guest id with their room-no. that are of company type?} \\\

\textbf{Ans-}		\textbf{SQL-}
\begin{lstlisting}
					SELECT 		 	company.guest_id, 
 		  								alloted.room_no
					FROM 				hotel.alloted, 
		  								hotel.company
					WHERE 			company.guest_id = alloted.guest_id;
\end{lstlisting}
\textbf{Output-} \\			   
\begin{figure}[hbtp]
\centering
\includegraphics[width=0.4\columnwidth]{fig33}
\caption{\textbf{{\color{red}Executed output table}}}
\end{figure}
\\\

\newpage
\textbf{Qu\_34-}  \textbf{List out the facility id used by guest that lived in room-no = A101?} \\\

\textbf{Ans-}		\textbf{SQL-}
\begin{lstlisting}
					SELECT 		alloted.room_no, 
  									uses.facility_id, 
	            				facility.facility_type
					FROM			hotel.uses, 
  									hotel.alloted, 
 									hotel.facility
					WHERE		uses.facility_id = facility.facility_id AND
  				alloted.guest_id = uses.guest_id AND alloted.room_no = 'A101'
\end{lstlisting}
\textbf{Output-} \\			   
\begin{figure}[hbtp]
\centering
\includegraphics[width=0.4\columnwidth]{fig34}
\caption{\textbf{{\color{red}Executed output table}}}
\end{figure}
\\\

\textbf{Qu\_35-}  \textbf{List out the date on which maximum number of customer of family tpye came to hotel?} \\\

\textbf{Ans-}		\textbf{SQL-}
\begin{lstlisting}
					DROP TABLE new2;
					CREATE TABLE new2 AS ( SELECT  
  							count(guest.guest_id) as count,
  							date(guest.entry_time) 
					FROM		hotel.guest, 
  								hotel.family
					WHERE 		family.guest_id = guest.guest_id
					GROUP BY		date(guest.entry_time) );
					DROP TABLE 		new3;
					CREATE TABLE 	new3 as ( SELECT	max(count) as count
														FROM	new2 );
					SELECT		*
					FROM	new2	NATURAL JOIN new3
\end{lstlisting}
\textbf{Output-} \\			   
\begin{figure}[hbtp]
\centering
\includegraphics[width=0.4\columnwidth]{fig35}
\caption{\textbf{{\color{red}Executed output table}}}
\end{figure}
\\\

\textbf{Qu\_36-}  \textbf{List out the date on which maximum number of customer of company tpye came to hotel?} \\\

\textbf{Ans-}		\textbf{SQL-}
\begin{lstlisting}
					DROP TABLE new2;
					CREATE TABLE new2 AS ( SELECT  
  							count(guest.guest_id) as count,
  							date(guest.entry_time) 
					FROM		hotel.guest, 
  								hotel.company
					WHERE 		company.guest_id = guest.guest_id
					GROUP BY		date(guest.entry_time) );
					DROP TABLE 		new3;
					CREATE TABLE 	new3 as ( SELECT	max(count) as count
														FROM	new2 );
					SELECT		*
					FROM	new2	NATURAL JOIN new3
\end{lstlisting}
\textbf{Output-} \\			   
\begin{figure}[hbtp]
\centering
\includegraphics[width=0.4\columnwidth]{fig36}
\caption{\textbf{{\color{red}Executed output table}}}
\end{figure}
\\\

\textbf{Qu\_37-}  \textbf{Find out the guest\_id that live the maximum number of days in the hotel?} \\\

\textbf{Ans-}		\textbf{SQL-}
\begin{lstlisting}
						DROP TABLE x1;
						CREATE TABLE x1 AS( SELECT alloted.guest_id,
							max(alloted.check_out_date-alloted.check_in_date) AS max
						FROM   hotel.alloted
						GROUP BY  alloted.guest_id	);
						DROP TABLE x2 ;
						CREATE TABLE x2 AS ( SELECT max(max) AS max from x1 );
						SELECT x1.guest_id,
									x1.max
  						FROM    x1 natural join x2
\end{lstlisting}
\textbf{Output-} \\			   
\begin{figure}[hbtp]
\centering
\includegraphics[width=0.4\columnwidth]{fig37}
\caption{\textbf{{\color{red}Executed output table}}}
\end{figure}

%%%%%%%%%%%%%%%%%%%%%%%%%%%%%%%% KJ END %%%%%%%%%%%%%%%%%%%%%%%%%%%%%%%%%%%





\newpage
\textbf{Qu\_38-}  \textbf{List names,guest\_id of families and company check\_in on 11-01-2015 ?} \\\

\textbf{Ans-}		\textbf{SQL-}
\begin{lstlisting}
						SELECT guest_id,Family_Head_name AS name 
						FROM family 
						WHERE guest_id IN (SELECT guest_id FROM Guest WHERE 	guest_id like '%F%' and Date(Entry_time)='2015-01-11') 
 						UNION 
						SELECT guest_id,Name AS name 
						FROM company 
						WHERE guest_id in (SELECT guest_id FROM Guest WHERE guest_id like '%C%' and Date(Entry_time)='2015-01-11');
\end{lstlisting}
\textbf{Output-} \\			   
\begin{figure}[hbtp]
\centering
\includegraphics[width=0.6\columnwidth]{fig38}
\caption{\textbf{{\color{red}Executed output table}}}
\end{figure}

\textbf{Qu\_39-}  \textbf{List names of different companies arrived from 7-1-2015 to 17-1-2015 ?} \\\

\textbf{Ans-}		\textbf{SQL-}
\begin{lstlisting}
						SELECT DISTINCT name 
						FROM company 
						WHERE guest_id IN ( SELECT guest_id FROM Allot WHERE guest_id like '%F%' and check_in_date >'2015-01-07' and check_out_date<='2015-01-15'); 
\end{lstlisting}
\textbf{Output-} \\			   
\begin{figure}[hbtp]
\centering
\includegraphics[width=1.0\columnwidth]{fig39}
\caption{\textbf{{\color{red}Executed output table}}}
\end{figure}

\newpage
\textbf{Qu\_40-}  \textbf{List guest\_id and name of family arrived on 24-1-2015 but not booked room  ?} \\\

\textbf{Ans-}		\textbf{SQL-}
\begin{lstlisting}
						SELECT  family_members.guest_id,
									family_members.Name  
						FROM 	family,family_members  
						WHERE  family_members.guest_id IN ((SELECT guest_id FROM 	bill WHERE payment_Date='2015-01-24' and guest_id like '%F%') 
						EXCEPT (SELECT Allot.guest_id FROM Allot
									WHERE Allot.guest_id like '%F%' and check_in_date='2015-01-24'));
\end{lstlisting}
\textbf{Output-} \\			   
\begin{figure}[hbtp]
\centering
\includegraphics[width=1.0\columnwidth]{fig40}
\caption{\textbf{{\color{red}Executed output table}}}
\end{figure}

\newpage
\subsection*{Embedded SQL in C:}
Embedded SQL language help in providing a interface that we can directly insert,update,delete data in the pgAdmin3.
\subsubsection*{Introduction}
 We all know that embedded sql is the way to connect our database with the pgAdmin, and we can manually insert, delete and update in the database using the user interface.\\\\
 
Here is an embedded sql code for the insertion, deletion and update for three table of our database:\\
1. Food \\
2. Facility \&\\
3. Room \\\\

We wanted to mention the attributes in our tables defined above.\\\\
1. Food -       \{Product\_Id\{PK\}, rate, type, name\}\\
2. Facility -   \{Facility\_Id\{PK\},rate, no\_of\_hours, facility\_type\}\\
3. Room    -    \{Room\_No\{PK\}, rate, type, status\_occupied \}\\\\
\begin{lstlisting}
#include<stdio.h>
#include<string.h>
#include<stdlib.h>
#include "sqlca.h"

EXEC SQL INCLUDE sqlca;

EXEC SQL BEGIN DECLARE SECTION;

// Keeping Facility Record
typedef struct
{
varchar f_id[50];
int f_rate;
int f_time;
varchar f_type[50];
} facility_record;

// Keeping Uses Record

typedef struct
{
varchar fa_id[50];
varchar guest_id[50];
int quantity;
} uses_record;

// Keeping Food Record
typedef struct
{
varchar food_id[50];
int food_rate;
varchar food_type[50];
varchar food_name[50];
} food_record;


// Keeping Room Record
typedef struct
{
varchar room_id[50];
varchar room_type[50];
int room_rate;
varchar room_status[50];
} room_record;

EXEC SQL END DECLARE SECTION;

////////////////////////////////////////////////////////////////////////////////////////////////////////////////////////////////////////////


void prompt(char s[], char t[]) {
char c;
int i = 0;
printf("%s",s);
while ((c = getchar()) != '\n') {
t[i] = c;
i++;
}
t[i] = '\0';
}

// Insert Item on Food

void insert_food()
{

printf("\n Enter details:\n");

EXEC SQL BEGIN DECLARE SECTION;
food_record fo;
EXEC SQL END DECLARE SECTION;

printf("Enter Product ID\n");
scanf("%s",fo.food_id.arr);
fo.food_id.len = strlen(fo.food_id.arr);

printf("Enter the food rate\n");
scanf("%d",&fo.food_rate);
getchar();

printf("Type of the Food\n ");
scanf("%s",fo.food_type.arr);
fo.food_type.len = strlen(fo.food_type.arr);

printf("Enter the Name of the Food");
scanf("%s",fo.food_name.arr);
fo.food_name.len = strlen(fo.food_name.arr);

EXEC SQL set transaction read write;
EXEC SQL insert into food values(:fo.food_id,:fo.food_rate,:fo.food_type,:fo.food_name);

if (sqlca.sqlcode < 0) {
printf("\n\nPRODUCT (%s) already exists\n",fo.food_id.arr);
EXEC SQL rollback work;
EXEC SQL commit;
return;
}
else
{
printf("\n New Food Item Added Succesfully\n");
EXEC SQL commit;
return;
}
}

////////////////////////////////////////////////////////////////////////////////////////////////////////////////////////////////////////////

//  Insert Item in Room

void insert_room()
{
printf("Enter the details of the new room\n");
EXEC SQL BEGIN DECLARE SECTION;
room_record ro;
EXEC SQL END DECLARE SECTION;

printf("\nEnter the Room_Number ");
scanf("%s",ro.room_id.arr);
ro.room_id.len=strlen(ro.room_id.arr);

printf("\nEnter the type of the room");
scanf("%s",ro.room_type.arr);
ro.room_type.len=strlen(ro.room_type.arr);

printf("\nEnter the rate of the room");
scanf("%d", &ro.room_rate);
getchar();

printf("\nEnter the Status");
scanf("%s",ro.room_status.arr);
ro.room_status.len=strlen(ro.room_status.arr);


EXEC SQL set transaction read write;
EXEC SQL insert into room values(:ro.room_id,:ro.room_type,:ro.room_rate,:ro.room_status);

if (sqlca.sqlcode < 0) {
printf("\n\nRoom (%s) is already added\n",ro.room_id.arr);
EXEC SQL rollback work;
return;
}
else 
{
printf("\n New Room Added Succesfully\n");
EXEC SQL commit;
return;
} 
}
////////////////////////////////////////////////////////////////////////////////////////////////////////////////////////////////////////


void insert_facility()
{
printf("Enter the details of the new facility\n");
EXEC SQL BEGIN DECLARE SECTION;
facility_record fo;
EXEC SQL END DECLARE SECTION;

printf("\nEnter Facility ID ");
scanf("%s",fo.f_id.arr);
fo.f_id.len=strlen(fo.f_id.arr);

printf("\nEnter the rate of the facility");
scanf("%d", &fo.f_rate);
getchar();

printf("\nEnter the no_of_hours of the facility");
scanf("%d", &fo.f_time);
getchar();

printf("\nEnter the type of the facility");
scanf("%s",fo.f_type.arr);
fo.f_type.len=strlen(fo.f_type.arr);

EXEC SQL set transaction read write;
EXEC SQL insert into facility values(:fo.f_id,:fo.f_rate,:fo.f_time,:fo.f_type);

if (sqlca.sqlcode < 0) {
printf("\n\nFaclity (%s) is already added\n",fo.f_id.arr);
EXEC SQL rollback work;
return;
}
else 
{
printf("\n New Faclity Added Succesfully\n");
EXEC SQL commit;
return;
} 
}
///////////////////////////////////////////////////////////////////////////////////////////////////////////////////////////////////////////

void remove_food() 
{
EXEC SQL begin declare section;
food_record frec;
varchar food_n[50];
int onum;
varchar pid[20];
varchar pname[20];
varchar ptype[20];
int prate;
EXEC SQL end declare section;

printf("Enter Product_ID to be deleted: ");
scanf("%s",food_n.arr);
food_n.len = strlen(food_n.arr);


EXEC SQL select *
into :frec
from food
where product_id = :food_n;

if (sqlca.sqlcode > 0) {
printf("Product ID does not exist\n");
return;
}

EXEC SQL declare del_cur cursor for
select product_id from food where product_id = :food_n;

EXEC SQL set transaction read only;
EXEC SQL open del_cur;
EXEC SQL fetch del_cur into :onum;
if (sqlca.sqlcode == 0) {
printf("food_item is in ordered so delete order first\n");
EXEC SQL commit;
return;
}

EXEC SQL commit;
EXEC SQL set transaction read write;
EXEC SQL delete from food where product_id = :food_n;
EXEC SQL DECLARE deleteS CURSOR FOR
            SELECT * FROM food WHERE product_id = :food_n;
 
   EXEC SQL OPEN deleteS;
 
   do {
      EXEC SQL FETCH deleteS INTO :pid, :prate, :ptype, :pname;
      if (sqlca.sqlcode != 0) 
       break;
      printf( "\nDeleting the product...\n");
      EXEC SQL DELETE FROM food WHERE CURRENT OF deleteS;
      } while ( 1 );
 
   EXEC SQL CLOSE deleteS;
printf("\nFood DELETED\n");
EXEC SQL commit;
}

/////////////////////////////////////////////////////////////////////////////////////////////////////////////////////////////////////////////

void remove_facility() 
{
EXEC SQL begin declare section;
facility_record farec;
uses_record urec;
varchar faci_n[50];
int onum;
varchar fid[20];
varchar ftype[20];
int ftime[20];
int frate;
EXEC SQL end declare section;

printf("Enter Facility_ID to be deleted: ");
scanf("%s",faci_n.arr);
faci_n.len = strlen(faci_n.arr);


EXEC SQL select *
into :farec
from facility
where facility_id = :faci_n;

if (sqlca.sqlcode > 0) {
printf("Facility ID  does not exist\n");
	return;

}


EXEC SQL declare del_cur1 cursor for
select facility_id from facility where facility_id = :faci_n;

EXEC SQL set transaction read only;
EXEC SQL open del_cur1;
EXEC SQL fetch del_cur1 into :onum;
if (sqlca.sqlcode == 0) {
printf("facility_id is in use so delete order first\n");
EXEC SQL commit;
return;
}

EXEC SQL commit;
EXEC SQL set transaction read write;
EXEC SQL delete from facility where facility_id = :faci_n;
EXEC SQL DECLARE deleteP CURSOR FOR
            SELECT * FROM facility WHERE facility_id = :faci_n;
 
   EXEC SQL OPEN deleteP;
 
   do {
      EXEC SQL FETCH deleteP INTO :fid, :frate, :ftime, :ftype;
      if (sqlca.sqlcode != 0) break;
      printf( "\nDeleting the facility id...\n");
      EXEC SQL DELETE FROM facility WHERE CURRENT OF deleteS;
      } while ( 1 );
 
   EXEC SQL CLOSE deleteP;
printf("\nFacility DELETED\n");
EXEC SQL commit;
}
////////////////////////////////////////////////////////////////////////////////////////////////////////////////////////////////////////////

void remove_room() 
{
EXEC SQL begin declare section;
room_record rrec;
varchar room_n[50];
int onum;
varchar rid[20];
varchar rtype[20];
varchar rstatus[20];
int rrate;
EXEC SQL end declare section;

printf("Enter Room_Number to be deleted: ");
scanf("%s",room_n.arr);
room_n.len = strlen(room_n.arr);


EXEC SQL select *
into :rrec
from room
where room_no = :room_n;

if (sqlca.sqlcode > 0) {
printf("Room does not exist\n");
return;
}


EXEC SQL declare del_cur2 cursor for
select room_no from room where room_no = :room_n;

EXEC SQL set transaction read only;
EXEC SQL open del_cur2;
EXEC SQL fetch del_cur2 into :onum;
if (sqlca.sqlcode == 0) {
printf("room_no is alloted first delete the allot\n");
EXEC SQL commit;
return;
}

EXEC SQL commit;
EXEC SQL set transaction read write;
EXEC SQL delete from room where room_no = :room_n;
EXEC SQL DECLARE deleteQ CURSOR FOR
            SELECT * FROM room facility WHERE room_no= :room_n;
 
   EXEC SQL OPEN deleteQ;
 
   do {
      EXEC SQL FETCH deleteQ INTO :rid, :rtype, :rrate, :rstatus;
      if (sqlca.sqlcode != 0) break;
      printf( "\nDeleting the given Room...\n");
      EXEC SQL DELETE FROM room WHERE CURRENT OF deleteS;
      } while ( 1 );
 
   EXEC SQL CLOSE deleteQ;
printf("\nRoom DELETED\n");
EXEC SQL commit;
}

///////////////////////////////////////////////////////////////////////////////////////////////////////////////////////////////////////////


void update_food() {

EXEC SQL SET SEARCH_PATH TO hotel;
EXEC SQL begin declare section;
food_record food1;
varchar fo_id[40];
varchar foo_id[40];
varchar n[30];
varchar t[30];
int r;
int temp;
EXEC SQL end declare section;

printf("Enter the product_id you want to update\n ");
scanf("%s",fo_id.arr);
fo_id.len = strlen(fo_id.arr);
EXEC SQL select *
into
:food1
from
food
where product_id = :fo_id;
if (sqlca.sqlcode > 0) {
printf("food_item(%s) does not exist\n",fo_id.arr);
return;
}

printf("Current rate: %d\n",food1.food_rate);
printf("Enter the rate :\n ");
scanf("%d", &r);
food1.food_rate = r;

printf("Current Tpye%s\n",food1.food_type.arr);
printf("New Type of the food : \n");
scanf("%s",t.arr);
if (strlen(t.arr) > 1) {
strcpy(food1.food_type.arr,t.arr);
food1.food_type.len = strlen(food1.food_type.arr);
}

printf("Current Name%s\n",food1.food_name.arr);
printf("New Name of the food :\n ");
scanf("%s", n.arr);
if (strlen(n.arr) > 1) {
strcpy(food1.food_name.arr,n.arr);
food1.food_name.len = strlen(food1.food_name.arr);
}



EXEC SQL set transaction read write; 
EXEC SQL UPDATE food
set product_id = :food1.food_id,
rate = :food1.food_rate,
type = :food1.food_type,
name = :food1.food_name
where product_id = :food1.food_id;
if (sqlca.sqlcode < 0) {
printf("\n\nError on Update\n");
EXEC SQL rollback work;
return;
}
EXEC SQL commit;
printf("\nFood Item (%s) updated.\n",food1.food_id.arr);
}

///////////////////////////////////////////////////////////////////////////////////////////////////////////////////////////////////////////

void update_room() {

EXEC SQL SET SEARCH_PATH TO hotel;
EXEC SQL begin declare section;
room_record room1;
varchar ro_id[40];
varchar roo_id[40];
varchar rs[30];
varchar t[30];
int r;
EXEC SQL end declare section;

printf("Enter the room_id you want to update \n");
scanf("%s",ro_id.arr);
ro_id.len = strlen(ro_id.arr);
EXEC SQL select *
into
:room1
from
room
where room_no = :ro_id;
if (sqlca.sqlcode > 0) {
printf("room(%s) does not exist\n",ro_id.arr);
return;
}

printf("Current rate: %d\n",room1.room_rate);
printf("Enter the rate:\n ");
scanf("%d", &r);
room1.room_rate = r;

printf("Current Tpye%s\n",room1.room_type.arr);
printf("New Type of the room:\n");
scanf("%s",t.arr);
if (strlen(t.arr) > 1) {
strcpy(room1.room_type.arr,t.arr);
room1.room_type.len = strlen(room1.room_type.arr);
}

printf("Current Status%s\n",room1.room_status.arr);
printf("New Status of the room: \n");
scanf("%s", rs.arr);
if (strlen(rs.arr) > 1) {
strcpy(room1.room_status.arr,rs.arr);
room1.room_status.len = strlen(room1.room_status.arr);
}



EXEC SQL set transaction read write; 
EXEC SQL UPDATE room
set room_no = :room1.room_id,
type = :room1.room_type,
rate = :room1.room_rate,
status_occupied = :room1.room_status
where room_no = :room1.room_id;
if (sqlca.sqlcode < 0) {
printf("\n\nError on Update\n");
EXEC SQL rollback work;
return;
}
EXEC SQL commit;
printf("\nRoom is (%s) updated.\n",room1.room_id.arr);
}

///////////////////////////////////////////////////////////////////////////////////////////////////////////////////////////////////////////

void update_facility() {

EXEC SQL SET SEARCH_PATH TO hotel;
EXEC SQL begin declare section;
facility_record fac1;
varchar fa_id[40];
varchar faa_id[40];
varchar ft[30];
int fh;
int r;
EXEC SQL end declare section;

printf("Enter the facility_id you want to update \n");
scanf("%s",fa_id.arr);
fa_id.len = strlen(fa_id.arr);
EXEC SQL select *
into
:fac1
from
facility
where facility_id = :fa_id;
if (sqlca.sqlcode > 0) {
printf("facility(%s) does not exist\n",fa_id.arr);
return;
}

printf("Current rate: of the facility %d\n",fac1.f_rate);
printf("Enter the rate:\n ");
scanf("%d", &r);
fac1.f_rate = r;

printf("Current Time for the facility%d\n",fac1.f_time);
printf("New Time for the facility\n");
scanf("%d",&fh);
fac1.f_time = fh;


printf("Current Type%s\n",fac1.f_type.arr);
printf("New Type of the facility: \n");
scanf("%s", ft.arr);
if (strlen(ft.arr) > 1) {
strcpy(fac1.f_type.arr,ft.arr);
fac1.f_type.len = strlen(fac1.f_type.arr);
}


EXEC SQL set transaction read write; 
EXEC SQL UPDATE facility
set facility_id = :fac1.f_id,
rate = :fac1.f_rate,
no_of_hours = :fac1.f_time,
facility_type = :fac1.f_type
where facility_id = :fac1.f_id;
if (sqlca.sqlcode < 0) {
printf("\n\nError on Update\n");
EXEC SQL rollback work;
return;
}
EXEC SQL commit;
printf("\nfacility is (%s) updated.\n",fac1.f_id.arr);
}



////////////////////////////////////////////////////////////////////////////////////////////////////////////////////////////////////////////

void main_menu()
{
printf("\n\n");
printf("\t\t\t\t\t\tWelcome to GUI For Our Project\n\n");
printf("\t\t\t*   *   * *****  *    ****  ****   *      *  *****    ***  ****    *  * **** *** **** *   \n");
printf("\t\t\t * * * *  ***    *    *     *  *   * *  * *  ***       *   *  *    **** *  *  *  ***  *   \n");
printf("\t\t\t  *   *   *****  **** ****  ****   *   *  *  *****     *   ****    *  * ****  *  **** ***  \n\n");


printf("\t\t\t\t\t--------------------Hotel Management System-----------------\n\n");
printf("\n\t\t\t\t\tPlease select an option:\n");
printf("\t\t\t\t\t1. Insert a new Entry\n");
printf("\t\t\t\t\t2. Update an existing entry\n");
printf("\t\t\t\t\t3. Remove an existing Entry\n");
printf("\t\t\t\t\t4. Exit\n\n");
printf("\t\t\t\t\t*********************************************************\n");
}

////////////////////////////////////////////////////////////////////////////////////////////////////////////////////////////////////////////

void main_insert()
{
int insert_choice;
printf("Enter the type of a new Entry\n");
printf("1. Food\n");
printf("2. Room\n");
printf("3. Facility\n");
printf("4. Exit\n");
printf("Enter your choice:  ");
scanf("%d",&insert_choice);
switch(insert_choice)
{
case 1:
insert_food();
break;
case 2:
insert_room();
break;
case 3:
insert_facility();
break;
case 4:
EXEC SQL COMMIT;
EXEC SQL DISCONNECT;
exit(0);
}
}

////////////////////////////////////////////////////////////////////////////////////////////////////////////////////////////////////////////


void main_remove()
{
int remove_choice;
printf("Enter the type of entry you want to delete\n");
printf("1. Food\n");
printf("2. Facility\n");
printf("3. Room\n");
printf("4. Exit\n");
printf("Enter your choice:  ");
scanf("%d",&remove_choice);
switch(remove_choice)
{
case 1:
remove_food();
break;
case 2:
remove_facility();
break;
case 3:
remove_room();
break;
case 4:
EXEC SQL COMMIT;
EXEC SQL DISCONNECT;
exit(0);
}
}


////////////////////////////////////////////////////////////////////////////////////////////////////////////////////////////////////////////

void main_update()
{
int remove_choice;
printf("Enter the type of entry you want to update\n");
printf("1. Food\n");
printf("2. Facility\n");
printf("3. Room\n");
printf("4. Exit\n");
printf("Enter your choice:  ");
scanf("%d",&remove_choice);
switch(remove_choice)
{
case 1:
update_food();
break;
case 2:
update_facility();
break;
case 3:
update_room();
break;
case 4:
EXEC SQL COMMIT;
EXEC SQL DISCONNECT;
exit(0);
}
}

///////////////////////////////////////////////////////////////////////////////////////////////////////////////////////////////////////////

int main()
{
EXEC SQL CONNECT TO try_project@localhost:5432 USER postgres;
EXEC SQL set search_path to hotel;;
int main_choice;
X:
main_menu();
printf("\nEnter your choice (eg. 1 for 1st choice):  ");
scanf("%d",&main_choice);
switch(main_choice)
{
case 1:
main_insert();
goto X;
break;
case 2:
main_update();
goto X;
break;
case 3:
main_remove();
goto X;
break;
case 4:
EXEC SQL COMMIT;
EXEC SQL DISCONNECT;
exit(0);
default:
printf("Re-enter your choice");
goto X;
}
EXEC SQL DISCONNECT try_project;
return 0;
}

\end{lstlisting}
\begin{figure}[hbtp]
\centering
\includegraphics[height = 0.7\columnwidth]{e}
\caption{\textbf{{\color{red} Sample output of Embedded sql Query}}}
\end{figure}

\section*{Conclusion:} HMS resolves almost all queries that a real world HMS should answer. It not only stores a huge database, but also manipulates, deletes, and updates data efficiently.

\section *{Refrences :}
\url {http://tinman.cs.gsu.edu/~raj/books/Oracle9-chapter-3.pdf} \\
\url{en.wikipedia.org}
\end{document}